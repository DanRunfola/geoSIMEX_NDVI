\section{Abstract}
Spatial imprecision exists in nearly all spatial information - from census data to satellite information, the exact geographic location to which a measurement can be attributed is rarely known.
Following this, there is a large and growing set of literature examining how different classes of models can integrate information on spatial imprecision in order to more accurately reflect available data.
Here, we present a flexible approach - geoSIMEX - which can provide parameter and error estimates for both linear and non-linear models while adjusting for spatial imprecision.
Further, this approach can provide an estimate of the potential value of exact spatial information by distinguishing between traditional model error and additional errors introduced by imprecision.
We illustrate this approach through a case study leveraging a novel, publically available dataset recording the location of Chinese aid in Africa.
Using a propensity-score based matching procedure, we integrate this information with information on nighttime lights and a number of other spatially explicit covariates to examine if there is evidence Chinese aid has caused an increase or decrease in luminosity.
We use these findings to argue for the importance of incorporating spatial uncertainty into analyses.

\textbf{Keywords}: SIMEX, Spatial Uncertainty, Spatial Data Integration, Simulation, Ecological Fallacy\\
